\section{Discussion}
The current setup for the proposed tool is only suitable for natural language inference task. However, due to the modular design and the many shared attribution among end-two-end NLP models, we can readily extend the system to handle other tasks. We plan to add support for more NLP task, such as neural machine translation, question and answer, in the future.
%
From our evaluation process, we found that the quality of automatic perturbed sentences can be a potential limitation of the tool.
Since we rely on WordNet to generate the perturbation, which has a rather inclusive definition for synonymous. Often, we can identify perturbed sentences that are not particularly meaningful (e.g., sentences with very obscured words or usages). 
Even if all the words are meaning, currently, there is no way we can verify whether the perturbed sentence is valid natural language composition or not.
%
However, from natural language process point of view the perturbation of sentence while maintaining semantic and sentence correctness is an open research area on its own. 
%As better method appears, we can easily swap out the sentence perturbation functionality 

To conclude, this work introduces the \textbf{NLIVZ}, a perturbation-driven visual interrogation system that provides the expert with a streamlined exploration environment for testing their hypothesis and obtain intuition about the model. The proposed system frees the researchers from interruptions and tedious operations and allows them to focus on the more productive activities.


%\begin{itemize}
%    \item how the visualization approaches generalizes to other NLP task
%    \item limitation
%    \item future direction
%\end{itemize}
