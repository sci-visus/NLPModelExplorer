\section{Background}
\label{sec:languageInference}

In order to full appreciate the technqiue discussed in this paper, certain background
in neural network and more specifically neural natural language process models are necessary.
In the following paragraphs, we aim to cover the most relevent backgound knowledge.

\shusen{high level NLP comments}

The grand challenge in Natural Language Processing (NLP) is to have machine to acquire
deep comprehension of textual information. Major tasks include Natural Language Inference,
Machine Translation, Questions Answering, Summarization, and etc.
Recently, neural networks models have performed strongly on these tasks.

Therefore, making sense and explaining predictions made by deep neural networks is not only
essential for validating and improving the model, but also becoming a necessity with
increasing demands for model accountability (e.g., what is the evidence for making the decision)
and model fairness (e.g., is the prediction affected by the bias in the training data).

% Recently, with the wide adoption of long-short term memory
% (LSTM) network and the introduction of attention mechanism,
% neural network based model have dominated nearly all linguistic tasks
% and thoroughtly refreshed many baseline performances.
% %
% However, the disruptive advance also brings enormous challenges.
% Netural network work based on has long been critizied for their opaque nature,
% and often been regarded as back box approach.
% Due to the opaque nature of the neural network model, interpret and making sense
% of many internal model mechanisms can be extremely challenging.

% \subsection{Neural Network Primer}


\subsection{Neural Network Model in NLP}

\shusen{What is the basic understanding of neural network What is the end-to-end}

% \subsection{Interprebilty Challenges}
Compared with Conputer Vision, the discrete nature of words and sentences
presents additional challenge for
interpreting the model, since many visualization technique often employed
for images rely on continuous nature of the input space (e.g., one can interpolate
real values much easier than interplolate between words/sentences).
%
On the other hand, a majority of recent NLP neural networks share the nature of
end-to-end model, where the entire model operates as a black-box that takes
vectorized input and yields final prediction for a specific task.
Therefore, to study the inside of end-to-end neural models, it is important to
explore the interaction between intermediate representations and predictions.

The introduction of attention mechanism~\cite{bahdanau2014neural} allows
pairwise interaction between hidden states. This interaction can be naturally explained
as a form of alignment which exposes an interpretable layer in end-to-end neural networks.
Recently attention has contributed to many strongly-performed NLP models
~\cite{parikh2016emnlp},~\cite{rush2015neural},~\cite{yang2016hierarchical},
~\cite{seo2016bidirectional},~\cite{schwartz2017high}.
We will focus on its application in Natural Language Inference task as proposed in~\cite{parikh2016emnlp}.
However notice that our visualization can be generalized to other attention models as well.

\subsection{Natural Language Inference}
Natural Language Inference(cite?) is an important machine understanding task in NLP.
The problem definition is to classify the relationship between a premise sentence and a hypothesis sentence.
The prediction must fall in one of three catergories: $\{Entailment, Contradiction, Neutral\}$.
$Entailment$ refers to that the textual information of hypothesis is embedded in premise;
$Contradiction$ means the textuals of hypothesis opposes that of premise;
and $Neutral$ implies no conclusion can be made on either $Entailment$ or $Contradiction$.

Recently ~\cite{BowmanAngeliPotts2015} proposed a large dataset for this task. We will examine
the decomposable attention model~\cite{parikh2016emnlp} on this dataset with perturbation.

% showing an example??

\shusen{What is the most standard example to illustrate NLI problem?}

% \begin{itemize}
%     \item what is the textaul entailment problem
%     \item the importance of textual entailment problem
%     \item how easily can the visualization method extends other NLP problem
% \end{itemize}
