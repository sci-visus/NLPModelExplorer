\subsection{Implementation}
\label{sec:implementation}
We design the tool with flexibility and extensibility in mind. One of the key objectives is to allow easy integration with existing NLP workflow. Therefore, instead of implementing the tool as a standalone application, we encapsulate the visualization functionality as a python library, in which the behavior of the visual interaction (e.g., once prediction button is pressed what should be the prediction result) is defined in python using existing model in the expert's workflow. Also, due to the separation between the visualization and the underlying model computation, we can readily adapt different models or the same model with different configurations or training. 

To create a visualization, users only need to import the library, create an instance of the visualization object, and specify a set of callback functions, such as generating a prediction, accessing attention, to link the visualization to their models. The code (consists only a few lines of code) use to create the visualization in this paper is included in appendix B.
%
The NLP model is implemented in python using \emph{pytorch}~\cite{PaszkeGrossChintala2017}.
The visual elements are implemented in javascript using \emph{D3.js} library , the python server act as glue between the javascript visualization and the \emph{pytorch} model.
